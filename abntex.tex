
\documentclass[
	% -- opções da classe memoir --
	article,			% indica que é um artigo acadêmico
	12pt,				% tamanho da fonte
	oneside,			% para impressão apenas no recto. Oposto a twoside
	a4paper,			% tamanho do papel. 
	% -- opções da classe abntex2 --
	%chapter=TITLE,		% títulos de capítulos convertidos em letras maiúsculas
	section=TITLE,		% títulos de seções convertidos em letras maiúsculas
	subsection=TITLE,	% títulos de subseções convertidos em letras maiúsculas
	%subsubsection=TITLE % títulos de subsubseções convertidos em letras maiúsculas
	% -- opções do pacote babel --
	english,			% idioma adicional para hifenização
	brazil,				% o último idioma é o principal do documento
	sumario=tradicional
	]{abntex2}


% ---
% PACOTES
% ---

% ---
% Pacotes fundamentais 
% ---
\usepackage{lmodern}			% Usa a fonte Latin Modern
\usepackage[T1]{fontenc}		% Selecao de codigos de fonte.
\usepackage[utf8]{inputenc}		% Codificacao do documento (conversão automática dos acentos)
\usepackage{indentfirst}		% Indenta o primeiro parágrafo de cada seção.
\usepackage{nomencl} 			% Lista de simbolos
\usepackage{color}				% Controle das cores
\usepackage{graphicx}			% Inclusão de gráficos
\usepackage{microtype} 			% para melhorias de justificação
% ---
		
% ---
% Pacotes adicionais, usados apenas no âmbito do Modelo Canônico do abnteX2
% ---
\usepackage{lipsum}				% para geração de dummy text
% ---
		
% ---
% Pacotes de citações
% ---
\usepackage[alf]{abntex2cite}	% Citações padrão ABNT
% ---


% --- Informações de dados para CAPA e FOLHA DE ROSTO ---
\titulo{Plugin de acessibilidade para sites institucionais da UFJF}
\tituloestrangeiro{Accessibility plugin for institutional websites of UFJF}

\autor{\texorpdfstring{
    João Victor Pereira dos Anjos
    \thanks{Graduando em Sistemas de Informação pela Universidade Federal de Juiz de Fora.}
    \and
    Gildo de Almeida Leonel
    }{}
}

\local{Brasil}
\data{2025}

% informações do PDF
\makeatletter
\hypersetup{
     	%pagebackref=true,
		pdftitle={\@title}, 
		pdfauthor={\@author},
    	pdfsubject={Plugin de acessibilidade para sites institucionais da UFJF},
	    pdfcreator={\LaTeX},
		pdfkeywords={acessibilidade}{plugin}{institucional}{ufjf}, 
		colorlinks=true,       		% false: boxed links; true: colored links
    	linkcolor=black,          	% color of internal links
    	citecolor=black,        		% color of links to bibliography
    	filecolor=black,      		% color of file links
		urlcolor=black,
		bookmarksdepth=4
}

\makeatother
% --- 

% ---
% compila o indice
% ---
\makeindex
% ---

% ---
% Altera as margens padrões
% ---
\setlrmarginsandblock{3cm}{3cm}{*}
\setulmarginsandblock{3cm}{3cm}{*}
\checkandfixthelayout
% ---

% --- 
% Espaçamentos entre linhas e parágrafos 
% --- 

% O tamanho do parágrafo é dado por:
\setlength{\parindent}{1.3cm}

% Controle do espaçamento entre um parágrafo e outro:
\setlength{\parskip}{0.2cm}  % tente também \onelineskip

% Espaçamento simples
\SingleSpacing


% ----
% Início do documento
% ----
\begin{document}

% Seleciona o idioma do documento (conforme pacotes do babel)
%\selectlanguage{english}
\selectlanguage{brazil}

% Retira espaço extra obsoleto entre as frases.
\frenchspacing

% ----------------------------------------------------------
% ELEMENTOS PRÉ-TEXTUAIS
% ----------------------------------------------------------

%---
%
% Se desejar escrever o artigo em duas colunas, descomente a linha abaixo
% e a linha com o texto ``FIM DE ARTIGO EM DUAS COLUNAS''.
% \twocolumn[    		% INICIO DE ARTIGO EM DUAS COLUNAS
%
%---

% página de titulo principal (obrigatório)
\maketitle


% titulo em outro idioma (opcional)



% resumo em português
\begin{resumoumacoluna}
    Conforme a ABNT NBR 6022:2018, o resumo no idioma do documento é elemento obrigatório.
    Constituído de uma sequência de frases concisas e objetivas e não de uma
    simples enumeração de tópicos, não ultrapassando 250 palavras, seguido, logo
    abaixo, das palavras representativas do conteúdo do trabalho, isto é,
    palavras-chave e/ou descritores, conforme a NBR 6028. (\ldots) As
    palavras-chave devem figurar logo abaixo do resumo, antecedidas da expressão
    Palavras-chave:, separadas entre si por ponto e finalizadas também por ponto.

    \vspace{\onelineskip}

    \noindent
    \textbf{Palavras-chave}: latex. abntex. editoração de texto.
\end{resumoumacoluna}


% resumo em inglês
\renewcommand{\resumoname}{Abstract}
\begin{resumoumacoluna}
    \begin{otherlanguage*}{english}
        According to ABNT NBR 6022:2018, an abstract in foreign language is optional.

        \vspace{\onelineskip}

        \noindent
        \textbf{Keywords}: latex. abntex.
    \end{otherlanguage*}
\end{resumoumacoluna}

% ]  				% FIM DE ARTIGO EM DUAS COLUNAS
% ---

\begin{center}\smaller
    \textbf{Data de submissão e aprovação}: elemento obrigatório. Indicar dia, mês e ano

    \textbf{Identificação e disponibilidade}: elemento opcional. Pode ser indicado
    o endereço eletrônico, DOI, suportes e outras informações relativas ao acesso.
\end{center}

% ----------------------------------------------------------
% ELEMENTOS TEXTUAIS
% ----------------------------------------------------------
\textual

% ----------------------------------------------------------
% Introdução
% ----------------------------------------------------------
\section{Introdução}

A palavra acessibilidade, sua origem etimológica é derivada do latim \textit{accessiblitas}
e significa ``condição para utilização, com segurança e autonomia,
total ou assistida, dos espaços, mobiliários e equipamentos urbanos, das
edificações, dos serviços de transporte e dos dispositivos, sistemas e meios de
comunicação e informação, por pessoa com deficiência ou mobilidade reduzida''\cite{CD}.

No Brasil, a acessibilidade é um direito garantido pela Constituição
Federal de 1988, pela Lei Brasileira de Inclusão (LBI) de 2015 \cite{LBI}
e por normas técnicas específicas, como a NBR 9050/2015 da Associação
Brasileira de Normas Técnicas~\cite{ABNT}. Essas legislações estabelecem
parâmetros para a promoção da acessibilidade em espaços públicos e privados,
visando a inclusão de pessoas com deficiência física, visual, auditiva,
intelectual e múltipla.

No âmbito digital a acessibilidade web é um pilar fundamental para a
inclusão, o WCAG 2.1/2.2 \cite{wcag22} sendo um conjunto de diretrizes
internacionais para a acessibilidade de conteúdo web, tem por objetivo,
tornar os sites mais acessíveis para as pessoas com deficiência visual, auditiva,
motora e cognitiva, garantindo a igualdade de acesso à informação e aos
serviços online. O eMag é um modelo nacional de acessibilidade em governo
eletrônico que estabelece diretrizes para a promoção da acessibilidade em
sites governamentais, com o objetivo de garantir a inclusão digital e o acesso
à informação para todos os cidadãos Brasileiros.

A Universidade Federal de Juiz de Fora (UFJF), como uma instituição pública,
gerência uma grande quantidade de sites e portais, que são regularmente
atualizados por diversas pessoas, como professores, pesquisadores, bolsistas e
servidores, da qual chamamos de conteudistas. A diversidade de conteúdos e
responsáveis torna o processo de garantia de acessibilidade primordial para
atender a legislação e promover a inclusão digital.

A fim de divulgar as informações referentes aos seus setores e atividades a
Universidade Federal de Juiz de Fora, UFJF, através do Centro de Gestão
do Conhecimento Organizacional (CGCO), é responsável por controlar a
disponibilização de sites, a padronização dos layouts e o suporte técnico. Para
a sustentação desse serviço, é utilizado o CMS WordPress~\cite{WP},
uma plataforma de gerenciamento de conteúdo que permite a criação e
manutenção de sites de forma simplificada e intuitiva.

Neste cenário o temos uma complexidade ao realizar auditorias manuais, que
consomem tempo e recursos, e principalmente a falta de ferramentas
centralizadas dentro da UFJF para aplicar o Modelo de Acessibilidade em
Governo Eletrônico~\cite{emag}, conhecido como eMag, e as Diretrizes
de Acessibilidade para Conteúdo Web~\cite{wcag22}, conhecidas como
WCAG, visto que esse documentos estabelecem parâmetros técnicos para essa inclusão.

Em uma análise preliminar avaliando a presença de elementos de acessibilidade nos
sites da UFJF, foi identificado que boa parte apresentavam falhas de acessibilidade, como imagens sem texto
alternativo e baixo contraste de cores, limitando o acesso de usuários com deficiência
visual. Além disso, a falta de padronização e de um processo de auditoria contínuo
dificulta a identificação e correção dessas falhas, comprometendo a qualidade e a
usabilidade dos sites.

Diante deste desafio, este artigo propõe uma solução técnica inovadora
para o contexto da UFJF, baseada em um plugin WordPress de auditoria
de acessibilidade que integra tecnologias modernas de automação,
análise técnica e processamento de dados. O sistema opera como um serviço
independente, com suporte a regras WCAG 2.1/2.2 e eMAG, permitindo
avaliações em tempo real e personalização de regras de acessibilidade.

Este documento e seu código-fonte são exemplos de referência de uso da classe
\textsf{abntex2} e do pacote \textsf{abntex2cite}. O documento exemplifica a
elaboração de publicação periódica científica impressa produzida conforme a ABNT
NBR 6022:2018 \emph{Informação e documentação - Artigo em publicação periódica
    científica - Apresentação}.

A expressão ``Modelo canônico'' é utilizada para indicar que \abnTeX\ não é
modelo específico de nenhuma universidade ou instituição, mas que implementa tão
somente os requisitos das normas da ABNT. Uma lista completa das normas
observadas pelo \abnTeX\ é apresentada em \citeonline{abntex2classe}.

Sinta-se convidado a participar do projeto \abnTeX! Acesse o site do projeto em
\url{http://www.abntex.net.br/}. Também fique livre para conhecer,
estudar, alterar e redistribuir o trabalho do \abnTeX, desde que os arquivos
modificados tenham seus nomes alterados e que os créditos sejam dados aos
autores originais, nos termos da ``The \LaTeX\ Project Public
License''\footnote{\url{http://www.latex-project.org/lppl.txt}}.

Encorajamos que sejam realizadas customizações específicas deste documento.
Porém, recomendamos que ao invés de se alterar diretamente os arquivos do
\abnTeX, distribua-se arquivos com as respectivas customizações. Isso permite
que futuras versões do \abnTeX~não se tornem automaticamente incompatíveis com
as customizações promovidas. Consulte \citeonline{abntex2-wiki-como-customizar}
para mais informações.

Este exemplo deve ser utilizado como complemento do manual da classe
\textsf{abntex2} \cite{abntex2classe}, dos manuais do pacote
\textsf{abntex2cite} \cite{abntex2cite,abntex2cite-alf} e do manual da classe
\textsf{memoir} \cite{memoir}. Consulte o \citeonline{abntex2modelo} para obter
exemplos e informações adicionais de uso de \abnTeX\ e de \LaTeX.

% ----------------------------------------------------------
% Seção de explicações
% ----------------------------------------------------------
\section{Exemplos de comandos}

\subsection{Margens}

A norma ABNT NBR 6022:2018 não estabelece uma margem específica a ser utilizada
no artigo científico. Dessa maneira, caso deseje alterar as margens, utilize os
comandos abaixo:

\begin{verbatim}
   \setlrmarginsandblock{3cm}{3cm}{*}
   \setulmarginsandblock{3cm}{3cm}{*}
   \checkandfixthelayout
\end{verbatim}

\subsection{Duas colunas}

É comum que artigos científicos sejam escritos em duas colunas. Para isso,
adicione a opção \texttt{twocolumn} à classe do documento, como no exemplo:

\begin{verbatim}
   \documentclass[article,11pt,oneside,a4paper,twocolumn]{abntex2}
\end{verbatim}

É possível indicar pontos do texto que se deseja manter em apenas uma coluna,
geralmente o título e os resumos. Os resumos em única coluna em documentos com
a opção \texttt{twocolumn} devem ser escritos no ambiente
\texttt{resumoumacoluna}:

\begin{verbatim}
   \twocolumn[              % INICIO DE ARTIGO EM DUAS COLUNAS

     \maketitle             % pagina de titulo

     \renewcommand{\resumoname}{Nome do resumo}
     \begin{resumoumacoluna}
        Texto do resumo.
      
        \vspace{\onelineskip}
 
        \noindent
        \textbf{Palavras-chave}: latex. abntex. editoração de texto.
     \end{resumoumacoluna}
   
   ]                        % FIM DE ARTIGO EM DUAS COLUNAS
\end{verbatim}

\subsection{Recuo do ambiente \texttt{citacao}}

Na produção de artigos (opção \texttt{article}), pode ser útil alterar o recuo
do ambiente \texttt{citacao}. Nesse caso, utilize o comando:

\begin{verbatim}
   \setlength{\ABNTEXcitacaorecuo}{1.8cm}
\end{verbatim}

Quando um documento é produzido com a opção \texttt{twocolumn}, a classe
\textsf{abntex2} automaticamente altera o recuo padrão de 4 cm, definido pela
ABNT NBR 10520:2002 seção 5.3, para 1.8 cm.

\section{Cabeçalhos e rodapés customizados}

Diferentes estilos de cabeçalhos e rodapés podem ser criados usando os
recursos padrões do \textsf{memoir}.

Um estilo próprio de cabeçalhos e rodapés pode ser diferente para páginas pares
e ímpares. Observe que a diferenciação entre páginas pares e ímpares só é
utilizada se a opção \texttt{twoside} da classe \textsf{abntex2} for utilizado.
Caso contrário, apenas o cabeçalho padrão da página par (\emph{even}) é usado.

Veja o exemplo abaixo cria um estilo chamado \texttt{meuestilo}. O código deve
ser inserido no preâmbulo do documento.

\begin{verbatim}
%%criar um novo estilo de cabeçalhos e rodapés
\makepagestyle{meuestilo}
  %%cabeçalhos
  \makeevenhead{meuestilo} %%pagina par
     {topo par à esquerda}
     {centro \thepage}
     {direita}
  \makeoddhead{meuestilo} %%pagina ímpar ou com oneside
     {topo ímpar/oneside à esquerda}
     {centro\thepage}
     {direita}
  \makeheadrule{meuestilo}{\textwidth}{\normalrulethickness} %linha
  %% rodapé
  \makeevenfoot{meuestilo}
     {rodapé par à esquerda} %%pagina par
     {centro \thepage}
     {direita} 
  \makeoddfoot{meuestilo} %%pagina ímpar ou com oneside
     {rodapé ímpar/onside à esquerda}
     {centro \thepage}
     {direita}
\end{verbatim}

Para usar o estilo criado, use o comando abaixo imediatamente após um dos
comandos de divisão do documento. Por exemplo:

\begin{verbatim}
   \begin{document}
     %%usar o estilo criado na primeira página do artigo:
     \pretextual
     \pagestyle{meuestilo}
     
     \maketitle
     ...
     
     %%usar o estilo criado nas páginas textuais
     \textual
     \pagestyle{meuestilo}
     
     \chapter{Novo capítulo}
     ...
   \end{document}  
\end{verbatim}

Outras informações sobre cabeçalhos e rodapés estão disponíveis na seção 7.3 do
manual do \textsf{memoir} \cite{memoir}.

\section{Mais exemplos no Modelo Canônico de Trabalhos Acadêmicos}

Este modelo de artigo é limitado em número de exemplos de comandos, pois são
apresentados exclusivamente comandos diretamente relacionados com a produção de
artigos.

Para exemplos adicionais de \abnTeX\ e \LaTeX, como inclusão de figuras,
fórmulas matemáticas, citações, e outros, consulte o documento
\citeonline{abntex2modelo}.

\section{Consulte o manual da classe \textsf{abntex2}}

Consulte o manual da classe \textsf{abntex2} \cite{abntex2classe} para uma
referência completa das macros e ambientes disponíveis.

% ---
% Finaliza a parte no bookmark do PDF, para que se inicie o bookmark na raiz
% ---
\bookmarksetup{startatroot}% 
% ---

% ---
% Conclusão
% ---
\section{Considerações finais}

\lipsum[1]

\begin{citacao}
    \lipsum[2]
\end{citacao}

\lipsum[3]

% ----------------------------------------------------------
% ELEMENTOS PÓS-TEXTUAIS
% ----------------------------------------------------------
\postextual

% ----------------------------------------------------------
% Referências bibliográficas
% ----------------------------------------------------------
\bibliography{abntex2-modelo-references}

% ----------------------------------------------------------
% Glossário
% ----------------------------------------------------------
%
% Há diversas soluções prontas para glossário em LaTeX. 
% Consulte o manual do abnTeX2 para obter sugestões.
%
%\glossary

% ----------------------------------------------------------
% Apêndices
% ----------------------------------------------------------

% ---
% Inicia os apêndices
% ---
\begin{apendicesenv}

    % ----------------------------------------------------------
    \chapter{Nullam elementum urna vel imperdiet sodales elit ipsum pharetra ligula
      ac pretium ante justo a nulla curabitur tristique arcu eu metus}
    % ----------------------------------------------------------
    \lipsum[55-56]

\end{apendicesenv}
% ---

% ----------------------------------------------------------
% Anexos
% ----------------------------------------------------------
\cftinserthook{toc}{AAA}
% ---
% Inicia os anexos
% ---
%\anexos
\begin{anexosenv}

    % ---
    \chapter{Cras non urna sed feugiat cum sociis natoque penatibus et magnis dis
      parturient montes nascetur ridiculus mus}
    % ---

    \lipsum[31]

\end{anexosenv}

% ----------------------------------------------------------
% Agradecimentos
% ----------------------------------------------------------

\section*{Agradecimentos}
Texto sucinto aprovado pelo periódico em que será publicado. Último
elemento pós-textual.

\end{document}