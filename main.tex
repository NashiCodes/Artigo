\documentclass[12pt]{article}
\usepackage{hyphenat}

\usepackage{xurl}
\usepackage[utf8]{inputenc}
\usepackage{csquotes}
\usepackage[brazil]{babel}
\usepackage{graphicx}
\usepackage[
    backend=biber,
    style=abnt,
    language=brazil,
    ]{biblatex}

\DeclareFieldFormat{url}{Disponível em\addcolon\space\url{#1}}
\addbibresource{references.bib}


\title{Plugin de acessibilidade para sites Governamentais}
\author{João Victor Pereira dos Anjos}
\date{Jan 2025}


\begin{document}

\maketitle
  
\tableofcontents

\section{Resumo}
   
This is the first section.
      
Lorem  ipsum  dolor  sit  amet,  consectetuer  adipiscing  
elit.   Etiam  lobortisfacilisis sem.  Nullam nec mi et 
neque pharetra sollicitudin.  Praesent imperdietmi nec ante. 
Donec ullamcorper, felis non sodales..

\section{Introdução}

A palavra acessibilidade, sua origem etimológica é derivada do 
latim \textit{accessiblitas} e significa ``condição para utilização, com segurança e
autonomia, total ou assistida, dos espaços, mobiliários e equipamentos urbanos, das 
edificações, dos serviços de transporte e dos dispositivos, sistemas e meios de 
comunicação e informação, por pessoa com deficiência ou mobilidade reduzida''\cite{Acessibilidade}.
No âmbito digital a acessibilidade web é um pilar fundamental para a inclusão, 
garantindo que todos os usuários, independentemente de suas capacidades físicas ou 
cognitivas, possam acessar, compreender e interagir com conteúdo online.\cite{wcag22}. 

Trazendo para o contexto brasileiro, o Modelo de Acessibilidade em Governo 
Eletrônico\cite{emag} e as Diretrizes de Acessibilidade para Conteúdo Web\cite{wcag22} 
estabelecem parâmetros técnicos para essa inclusão. No entanto, a implementação 
prática dessas diretrizes enfrenta desafios significativos, especialmente em 
instituições públicas de grande porte, como a Universidade Federal de Juiz de Fora 
(UFJF), que gerencia centenas de sites institucionais atualizados frequentemente por 
conteudistas sem formação técnica emacessibilidade.

O problema central reside na complexidade de auditorias manuais, que consomem
tempo e recursos, e na falta de ferramentas centralizadas dentro da UFJF para
identificar e corrigir violações. Por exemplo, em uma análise preliminar, boa parte 
das páginas e posts da UFJF apresentavam erros críticos, como imagens sem texto
alternativo e baixo contraste de cores, limitando o acesso de usuários com deficiência
visual. Além disso, a ausência de relatórios contextualizados dificultava a 
priorização de correções por parte das equipes responsáveis.

Neste cenário, este artigo propõe uma solução técnica inovadora: um sistema
automatizado de auditoria, integrado ao WordPress (plataforma amplamente utilizada
pela instituição), que combina:

\begin{itemize}
\item \textbf{Renderização headless da pagina via Puppeteer} para capturar conteúdo
dinâmico (ex: Single-Page Applications).\cite{puppeteer}
\item \textbf{Análise técnica com Axe-core\cite{axecore}}, cobrindo mais de 200 regras WCAG 2.1/2.2 e eMAG.\@
\item \textbf{APIs REST personalizadas} para avaliação de URLs internas, externas e
conteúdo HTML/CSS bruto.
\item \textbf{Pós-processamento inteligente} de relatórios, filtrando falsos
positivos e oferecer sugestões de correção em linguagem acessível.
\end{itemize}

A ferramenta não apenas otimiza processos técnicos, mas \textbf{democratiza a
fiscalização de acessibilidade}, empoderando conteudistas não especialistas com
dados claros e acionáveis. Este trabalho visa, portanto, contribuir para o debate 
sobre automação e inclusão digital, oferecendo um modelo replicável para instituições
públicas e privadas que buscam alinhar-se às exigências legais e éticas da
acessibilidade web.

\subsection{Contexto Institucional e Desafios Operacionais}\label{subsec:contexto}
A Universidade Federal de Juiz de Fora (UFJF) gerencia uma ampla rede de sites
institucionais, abrangendo portais acadêmicos, departamentais e projetos de extensão.
Esses sites são atualizados frequentemente por \textbf{conteudistas sem formação 
técnica em acessibilidade}, como professores, servidores administrativos e
colaboradores externos. A ausência de ferramentas especializadas para auditorias 
automatizadas resultava em desafios críticos:

\begin{itemize}
\item \textbf{Falta de padronização}: Conteúdos publicados sem verificação prévia de
elementos essenciais (ex: texto alternativo em imagens, rótulos em formulários).
\item \textbf{Dependência de auditorias manuais:} Processos demorados e suscetíveis 
a inconsistências, especialmente em páginas com conteúdo 
dinâ\-mico.
\item \textbf{Dificuldade de escalabilidade}: Impossibilidade de avaliar centenas de páginas de forma ágil e integrada à plataforma WordPress, amplamente utilizada pela
instituição.
\end{itemize}

Além disso, a natureza descentralizada da gestão dos sites gerava relatórios
fragmentados, dificultando a priorização de correções e a padronização de práticas de
acessibilidade. A falta de dados claros e contextualizados limitava a eficácia das
ações de capacitação e sensibilização promovidas pela UFJF, que visam promover uma
cultura inclusiva e acessível.

\subsection{Diretrizes e Fundamentação Técnica}\label{subsec:diretrizes}
O sistema foi estruturado com base em diretrizes reconhecidas internacionalmente e
adaptações para o contexto institucional:

\begin{enumerate}
    \item WCAG 2.1/2.2 (Níveis A e AA)
    \begin{itemize}
        \item Princípios fundamentais: 
        \begin{itemize}
            \item \textbf{Perceptível:} Garantia de alternativas textuais para mídias não
            textuais, contraste de cores adequado e organização lógica do conteúdo.
            \item \textbf{Operável:} Compatibilidade com navegação por teclado e tempo
            adequado para interação.
            \item \textbf{Compreensível:} Clareza na apresentação de informações e
            prevenção de erros de entrada.
            \item \textbf{Robusto:} Compatibilidade com tecnologias assistivas e
            manutenção de conteúdo acessível em atualizações.
        \end{itemize}
    \end{itemize}
    \item eMAG (Modelo de Acessibilidade em Governo Eletrônico)
    \begin{itemize}
        \item Especificidades para o setor público:
        \begin{itemize}
            \item Foco em formulários simplificados e linguagem acessível.
            \item Adaptação de termos técnicos para o português brasileiro.
            \item Recomendações para contraste visual adequado em interfaces
            institucionais.
        \end{itemize}
    \end{itemize}
    \item Critérios de Implementação do Sistema
    \begin{itemize}
        \item \textbf{Abrangência:} Suporte a regras de acessibilidade mapeadas pelo Axe-core,
        incluindo verificações para elementos críticos como imagens, formulários e
        estrutura semântica.
        \item \textbf{Flexibilidade:} Personalização via variáveis de ambiente, permitindo ajustes
        como desabilitar regras irrelevantes para contextos específicos (ex: sites
        estáticos sem formulários).
        \item \textbf{Integração:} Compatibilidade nativa com a API REST do WordPress, facilitando
        a adoção pelos gestores de conteúdo da UFJF.\@
    \end{itemize}
\end{enumerate}

\section{Arquitetura do Sistema}\label{sec:arquitetura}
O sistema proposto foi desenvolvido com uma arquitetura modular, dividida em
camadas interdependentes que garantem escalabilidade, segurança e eficiência na
avaliação de acessibilidade. A estrutura integra tecnologias modernas de automação,
análise técnica e processamento de dados, adaptadas às necessidades específicas da
UFJF.\@

\subsection{Componentes Técnicos e Funcionalidades}\label{subsec:componentes}
A arquitetura é sustentada por três componentes principais, cada um com funções
específicas:

\begin{itemize}
\item Puppeteer para Renderização Headless\cite{puppeteer}
\begin{itemize}
    \item \textbf{Objetivo:} Capturar conteúdo dinâmico de páginas web, como Single-Page
    Applications (SPAs) e elementos carregados via JavaScript.
    \item \textbf{Funcionalidades:}
    \begin{itemize}
        \item Navegação automatizada com suporte a eventos load e networkidle0 para
        garantir renderização completa.
        \item Configuração de timeout estendido para páginas complexas.\@
        \item Reutilização de instâncias do navegador para otimização de recursos.
    \end{itemize}
\end{itemize}
\item Axe-core para Análise de Acessibilidade\cite{axecore} (203 regras WCAG)
\begin{itemize}
    \item \textbf{Cobertura:} 203 regras WCAG 2.1/2.2 e eMAG\cite{emag,wcag22}, categorizadas por criticidade
    (crítico, sério, moderado).
    \item \textbf{Personalização:} 
    \begin{itemize}
        \item Filtragem de regras via variáveis de ambiente.
        \item Foco em elementos específicos (ex: \#conteudo-main em sites
        WordPress).
        \item Suporte a localização em português brasileiro.
    \end{itemize}
\end{itemize}
\item Módulo de Pós-Processamento de Relatórios.

\textbf{Etapas:}
    \begin{enumerate}
        \item \textbf{Filtragem de Falsos Positivos:} Remoção de imagens com texto alternativo vazio
        mas declaradas no elemento HTML (alt=`` '') das avaliações que passaram 
        (consideradas intencionalmente decorativas).
        \item \textbf{Contextualização de Erros:} Limpa propriedades desnecessá\-rias e erros
        duplicados.
        \item \textbf{Estruturação de Relatórios:} Agregação de metadados como timestamp, URL e
        contagem de violações por tipo.
    \end{enumerate}
\end{itemize}

\subsection{Integração com WordPress e APIs REST}\label{subsec:api}
O sistema opera como um plugin WordPress com backend independente, garantindo
flexibilidade e segurança. A integração com a API REST nativa do WordPress permite
avaliações de acessibilidade em tempo real. As principais funcionalidades incluem:

\begin{itemize}
\item API REST Personalizada
\begin{itemize}
    \item \textbf{Endpoints Principais:}
    \begin{itemize}
        \item \textbf{POST /login: }Autenticação via Basic Auth, retornando token JWT válido
        por 1 hora.
        \item \textbf{GET /internal/site/{id}: }Avaliação completa de um site
        WordPress (todos os posts).
        \item \textbf{GET internal/site/1/post/?url={urlDoPost}: } Análise de um
        post específico de um site específico.
    \end{itemize}
    \item \textbf{Autenticação e Segurança:}
    \begin{itemize}
        \item Tokens JWT assinados com chave privada armazenada em variáveis de
        ambiente.
        \item Middleware de validação em todas as rotas protegidas.
        \item \textbf{Por que JWT?}
        \begin{itemize}
            \item \textbf{Stateless:} Não requer armazenamento de sessão no servidor.
            \item \textbf{Seguro:} Assinatura digital previne tampering.
        \end{itemize}
    \end{itemize}
\end{itemize}
\item Fluxo de Trabalho no WordPress
\begin{itemize}
    \item \textbf{Frontend (Plugin WordPress):}
    \begin{itemize}
        \item Interface intuitiva para seleção de sites/posts a serem avaliados.
        \item Exibição de relatórios com sugestões de correção em cards interativos.
    \end{itemize}
\end{itemize}

\end{itemize}

\nocite{*}
\printbibliography[title={Referências}]

\end{document}