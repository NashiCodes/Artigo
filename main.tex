\documentclass[12pt]{article}
\usepackage[utf8]{inputenc}
\usepackage{csquotes}
\usepackage[brazil]{babel}
\usepackage{graphicx}
\usepackage[
    backend=biber,
    style=abnt,
    language=brazil,
    ]{biblatex}

\addbibresource{references.bib}


\title{Plugin de acessibilidade para sites Governamentais}
\author{João Victor Pereira dos Anjos}
\date{Jan 2025}


\begin{document}

\maketitle
  
\tableofcontents

\section{Resumo}
   
This is the first section.
      
Lorem  ipsum  dolor  sit  amet,  consectetuer  adipiscing  
elit.   Etiam  lobortisfacilisis sem.  Nullam nec mi et 
neque pharetra sollicitudin.  Praesent imperdietmi nec ante. 
Donec ullamcorper, felis non sodales..

\section{Introdução}

A palavra acessibilidade, sua origem etimológica é derivada do latim \textit{accessiblitas} 
e significa ``condição para utilização, com segurança e autonomia, 
total ou assistida, dos espaços, mobiliários e equipamentos urbanos, das 
edificações, dos serviços de transporte e dos dispositivos, sistemas e meios de
comunicação e informação, por pessoa com deficiência ou mobilidade reduzida''\cite{Acessibilidade}.

No Brasil, a acessibilidade é um direito garantido pela Constituição 
Federal de 1988, pela Lei Brasileira de Inclusão (LBI) de 2015 \Cite{LBI}
e por normas técnicas específicas, como a NBR 9050/2015 da Associação 
Brasileira de Normas Técnicas~\cite{ABNT}. Essas legislações estabelecem 
parâmetros para a promoção da acessibilidade em espaços públicos e privados,
visando a inclusão de pessoas com deficiência física, visual, auditiva, 
intelectual e múltipla.

No âmbito digital a acessibilidade web é um pilar fundamental para a 
inclusão, garantindo que todos os usuários, independentemente de suas
capacidades físicas ou cognitivas, possam acessar, compreender e interagir com
conteúdo online.\cite{wcag22}. 

A Universidade Federal de Juiz de Fora (UFJF), gerencia centenas 
de sites institucionais atualizados frequentemente por pessoas denominadas
conteudistas. 

O problema central reside na complexidade de auditorias manuais, que consomem
tempo e recursos, e na falta de ferramentas centralizadas dentro da UFJF para
aplicar o Modelo de Acessibilidade em 
Governo Eletrônico~\cite{emag}, conhecido como eMag, e as Diretrizes 
de Acessibilidade para Conteúdo Web~\cite{wcag22}, conhecidas como 
WCAG, visto que esse documentos estabelecem parâmetros técnicos para essa inclusão.

Em uma análise preliminar avaliando os sites institucionais da UFJF, foi
identificado que boa parte apresentavam falhas de acessibilidade, como imagens sem texto
alternativo e baixo contraste de cores, limitando o acesso de usuários com deficiência
visual. Além disso, a ausência de relatórios contextualizados dificultava a 
identificação e correção por parte dos conteúdistas.

Neste cenário, este artigo propõe uma solução técnica inovadora para o contexto
da UFJF, baseada em um plugin WordPress de acessibilidade que integra tecnologias
modernas de automação, análise técnica e processamento de dados. O sistema opera
como um serviço independente, com suporte a regras WCAG 2.1/2.2 e eMAG, e é
compatível com a API REST do WordPress, permitindo avaliações em tempo real e
personalização de regras de acessibilidade.

A ferramenta não apenas otimiza processos técnicos, mas democratiza a
fiscalização de acessibilidade, empoderando conteudistas não especialistas com
dados claros e acionáveis. Este trabalho visa, portanto, contribuir para o debate 
sobre automação e inclusão digital, sugerindo um modelo replicável para instituições
públicas e privadas que buscam alinhar-se às exigências legais e éticas da
acessibilidade web, dentro do escopo do eMag, WCAG 2.1/2.2 e o contexto de cada
organização.

\subsection{Contexto Institucional e Desafios Operacionais}\label{subsec:contexto}
A Universidade Federal de Juiz de Fora (UFJF) gerencia uma ampla rede de sites
institucionais, abrangendo portais acadêmicos, departamentais e projetos de extensão.
Esses sites são atualizados frequentemente por conteudistas muitas vezes\textbf{ sem formação 
técnica em acessibilidade estutura HTML}, como professores, servidores administrativos e
colaboradores externos, neste cenário, a ausência de ferramentas especializadas para auditorias 
automatizadas de acessibilidade disponíveis na UFJF impossibilitava a cobrança efetiva de
práticas inclusivas e a identificação de violações em larga escala. Além disso, se tornava
maçante e ineficiente a realização de auditorias manuais, que consumiam tempo e recursos
preciosos das equipes de TI e Comunicação. Como resultado, a UFJF enfrentava os seguintes
desafios operacionais: 

\begin{itemize}
\item \textbf{Falta de padronização}: Conteúdos publicados sem verificação prévia de
elementos essenciais (ex: texto alternativo em imagens, rótulos em formulários).
\item \textbf{Dependência de auditorias manuais:} Processos demorados e suscetíveis 
a inconsistências, especialmente em páginas com conteúdo 
dinâ\-mico.
\item \textbf{Dificuldade de escalabilidade}: Impossibilidade de avaliar centenas de 
páginas de forma ágil e integrada à plataforma WordPress, amplamente utilizada pela
instituição.
\item \textbf{Falta de dados contextualizados}: Relatórios fragmentados e sem
metadados relevantes, dificultando a priorização de correções e a sensibilização de
conteudistas.
\end{itemize}

Além disso, a falta de dados claros e contextualizados limitava a eficácia das
ações de capacitação e sensibilização promovidas pela UFJF, que visam promover uma
cultura inclusiva e acessível.

\subsection{Diretrizes e Fundamentação Técnica}\label{subsec:diretrizes}
O sistema foi estruturado com base em diretrizes reconhecidas internacionalmente e
adaptações para o contexto institucional:

\begin{enumerate}
    \item WCAG 2.1/2.2 (Níveis A e AA)
    \begin{itemize}
        \item Princípios fundamentais: 
        \begin{itemize}
            \item \textbf{Perceptível:} Garantia de alternativas textuais para mídias não
            textuais, contraste de cores adequado e organização lógica do conteúdo.
            \item \textbf{Operável:} Compatibilidade com navegação por teclado e tempo
            adequado para interação.
            \item \textbf{Compreensível:} Clareza na apresentação de informações e
            prevenção de erros de entrada.
            \item \textbf{Robusto:} Compatibilidade com tecnologias assistivas e
            manutenção de conteúdo acessível em atualizações.
        \end{itemize}
    \end{itemize}
    \item eMAG (Modelo de Acessibilidade em Governo Eletrônico)
    \begin{itemize}
        \item Especificidades para o setor público:
        \begin{itemize}
            \item Foco em formulários simplificados e linguagem acessível.
            \item Adaptação de termos técnicos para o português brasileiro.
            \item Recomendações para contraste visual adequado em interfaces
            institucionais.
        \end{itemize}
    \end{itemize}
    \item Critérios de Implementação do Sistema
    \begin{itemize}
        \item \textbf{Abrangência:} Suporte a regras de acessibilidade mapeadas pelo Axe-core,
        incluindo verificações para elementos críticos como imagens, formulários e
        estrutura semântica.
        \item \textbf{Flexibilidade:} Personalização via variáveis de ambiente, permitindo ajustes
        como desabilitar regras irrelevantes para contextos específicos (ex: sites
        estáticos sem formulários).
        \item \textbf{Integração:} Compatibilidade nativa com a API REST do WordPress, facilitando
        a adoção pelos gestores de conteúdo da UFJF.\@
    \end{itemize}
\end{enumerate}

\section{Arquitetura do Sistema}\label{sec:arquitetura}
O sistema proposto foi desenvolvido com uma arquitetura modular, dividida em
camadas interdependentes que garantem escalabilidade, segurança e eficiência na
avaliação de acessibilidade. A estrutura integra tecnologias modernas de automação,
análise técnica e processamento de dados, adaptadas às necessidades específicas da
UFJF.\@

\subsection{Componentes Técnicos e Funcionalidades}\label{subsec:componentes}
A arquitetura é sustentada por três componentes principais, cada um com funções
específicas:

\begin{itemize}
\item Puppeteer para Renderização Headless\cite{puppeteer}
\begin{itemize}
    \item \textbf{Objetivo:} Capturar conteúdo dinâmico de páginas web, como Single-Page
    Applications (SPAs) e elementos carregados via JavaScript.
    \item \textbf{Funcionalidades:}
    \begin{itemize}
        \item Navegação automatizada com suporte a eventos load e networkidle0 para
        garantir renderização completa.
        \item Configuração de timeout estendido para páginas complexas.\@
        \item Reutilização de instâncias do navegador para otimização de recursos.
    \end{itemize}
\end{itemize}
\item Axe-core para Análise de Acessibilidade\cite{axecore} (203 regras WCAG)
\begin{itemize}
    \item \textbf{Cobertura:} 203 regras WCAG 2.1/2.2 e eMAG\cite{emag,wcag22}, categorizadas por criticidade
    (crítico, sério, moderado).
    \item \textbf{Personalização:} 
    \begin{itemize}
        \item Filtragem de regras via variáveis de ambiente.
        \item Foco em elementos específicos (ex: \#conteudo-main em sites
        WordPress).
        \item Suporte a localização em português brasileiro.
    \end{itemize}
\end{itemize}
\item Módulo de Pós-Processamento de Relatórios.

\textbf{Etapas:}
    \begin{enumerate}
        \item \textbf{Filtragem de Falsos Positivos:} Remoção de imagens com texto alternativo vazio
        mas declaradas no elemento HTML (alt=`` '') das avaliações que passaram 
        (consideradas intencionalmente decorativas).
        \item \textbf{Contextualização de Erros:} Limpa propriedades desnecessá\-rias e erros
        duplicados.
        \item \textbf{Estruturação de Relatórios:} Agregação de metadados como timestamp, URL e
        contagem de violações por tipo.
    \end{enumerate}
\end{itemize}

\subsection{Integração com WordPress e APIs REST}\label{subsec:api}
O sistema opera como um plugin WordPress com backend independente, garantindo
flexibilidade e segurança. A integração com a API REST nativa do WordPress permite
avaliações de acessibilidade em tempo real. As principais funcionalidades incluem:

\begin{itemize}
\item API REST Personalizada
\begin{itemize}
    \item \textbf{Endpoints Principais:}
    \begin{itemize}
        \item \textbf{POST /login: }Autenticação via Basic Auth, retornando token JWT válido
        por 1 hora.
        \item \textbf{GET /internal/site/{id}: }Avaliação completa de um site
        WordPress (todos os posts).
        \item \textbf{GET internal/site/1/post/?url={urlDoPost}: } Análise de um
        post específico de um site específico.
    \end{itemize}
    \item \textbf{Autenticação e Segurança:}
    \begin{itemize}
        \item Tokens JWT assinados com chave privada armazenada em variáveis de
        ambiente.
        \item Middleware de validação em todas as rotas protegidas.
        \item \textbf{Por que JWT?}
        \begin{itemize}
            \item \textbf{Stateless:} Não requer armazenamento de sessão no servidor.
            \item \textbf{Seguro:} Assinatura digital previne tampering.
        \end{itemize}
    \end{itemize}
\end{itemize}
\item Fluxo de Trabalho no WordPress
\begin{itemize}
    \item \textbf{Frontend (Plugin WordPress):}
    \begin{itemize}
        \item Interface intuitiva para seleção de sites/posts a serem avaliados.
        \item Exibição de relatórios com sugestões de correção em cards interativos.
    \end{itemize}
\end{itemize}

\end{itemize}

\nocite{*}
\printbibliography[title={Referências}]

\end{document}