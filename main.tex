\documentclass[12pt,letterpaper]{article}
\usepackage[utf8]{inputenc}
\usepackage{csquotes}
\usepackage[brazil]{babel}
\usepackage{graphicx}
\usepackage[
    backend=biber,
    style=alphabetic,
    sorting=ynt
    ]{biblatex}

\addbibresource{refs.bib}
    
\title{Plugin de acessibilidade para sites Governamentais}
\author{João Victor Pereira dos Anjos}
\date{Jan 2025}


\begin{document}

\maketitle
  
\tableofcontents

\section{Resumo}
   
This is the first section.
      
Lorem  ipsum  dolor  sit  amet,  consectetuer  adipiscing  
elit.   Etiam  lobortisfacilisis sem.  Nullam nec mi et 
neque pharetra sollicitudin.  Praesent imperdietmi nec ante. 
Donec ullamcorper, felis non sodales..

\section{Introdução}

A palavra acessibilidade, sua origem etimológica é derivada do latim \textit{accessiblitas} e significacondição para utilização, com segurança e autonomia, total ou assistida, dos espaços, mobiliários e equipamentos urbanos, das edificações, dos serviços de transporte e dos dispositivos, sistemas e meios de comunicação e informação, por pessoa com deficiência ou mobilidade reduzida
       
\section*{Unnumbered Section}
\addcontentsline{toc}{section}{Unnumbered Section}

Lorem ipsum dolor sit amet, consectetuer adipiscing elit.  
Etiam lobortis facilisissem.  Nullam nec mi et neque pharetra 
sollicitudin.  Praesent imperdiet mi necante\ldots

\section{Second Section}
       
Lorem ipsum dolor sit amet, consectetuer adipiscing elit.  
Etiam lobortis facilisissem.  Nullam nec mi et neque pharetra 
sollicitudin.  Praesent imperdiet mi necante\ldots

\medskip

\printbibliography

\end{document}