\documentclass[12pt]{article}
\usepackage[utf8]{inputenc}
\usepackage{csquotes}
\usepackage{hyphenat}
\usepackage[brazil]{babel}
\usepackage{graphicx}
\usepackage[
    backend=biber,
    style=abnt,
    language=brazil,
    ]{biblatex}

\addbibresource{references.bib}


\title{Plugin de acessibilidade para sites institucionais da UFJF}
\author{João Victor Pereira dos Anjos}
\date{Jan 2025}


\begin{document}

\maketitle

\tableofcontents

\section{Resumo}

\section{Resumo}

This is the first section.

Lorem  ipsum  dolor  sit  amet,  consectetuer  adipiscing
elit.   Etiam  lobortisfacilisis sem.  Nullam nec mi et
neque pharetra sollicitudin.  Praesent imperdietmi nec ante.
Donec ullamcorper, felis non sodales..

\section{Introdução}

A palavra acessibilidade, sua origem etimológica é derivada do latim \textit{accessiblitas}
e significa ``condição para utilização, com segurança e autonomia,
total ou assistida, dos espaços, mobiliários e equipamentos urbanos, das
edificações, dos serviços de transporte e dos dispositivos, sistemas e meios de
comunicação e informação, por pessoa com deficiência ou mobilidade reduzida''\cite{Acessibilidade}.

No Brasil, a acessibilidade é um direito garantido pela Constituição
Federal de 1988, pela Lei Brasileira de Inclusão (LBI) de 2015 \Cite{LBI}
e por normas técnicas específicas, como a NBR 9050/2015 da Associação
Brasileira de Normas Técnicas~\cite{ABNT}. Essas legislações estabelecem
parâmetros para a promoção da acessibilidade em espaços públicos e privados,
visando a inclusão de pessoas com deficiência física, visual, auditiva,
intelectual e múltipla.

No âmbito digital a acessibilidade web é um pilar fundamental para a
inclusão, o WCAG 2.1/2.2 \autocite{wcag22} sendo um conjunto de diretrizes
internacionais para a acessibilidade de conteúdo web, tem por objetivo,
tornar os sites mais acessíveis para as pessoas com deficiência visual, auditiva,
motora e cognitiva, garantindo a igualdade de acesso à informação e aos
serviços online. O eMag é um modelo nacional de acessibilidade em governo
eletrônico que estabelece diretrizes para a promoção da acessibilidade em
sites governamentais, com o objetivo de garantir a inclusão digital e o acesso
à informação para todos os cidadãos Brasileiros.

A Universidade Federal de Juiz de Fora (UFJF), como uma instituição pública,
gerência uma grande quantidade de sites e portais, que são regularmente
atualizados por diversas pessoas, como professores, pesquisadores, bolsistas e
servidores, da qual chamamos de conteudistas. A diversidade de conteúdos e
responsáveis torna o processo de garantia de acessibilidade primordial para
atender a legislação e promover a inclusão digital.

A fim de divulgar as informações referentes aos seus setores e atividades a
Universidade Federal de Juiz de Fora, UFJF, através do Centro de Gestão
do Conhecimento Organizacional (CGCO), é responsável por controlar a
disponibilização de sites, a padronização dos layouts e o suporte técnico. Para
a sustentação desse serviço, é utilizado o CMS WordPress~\autocite{WP},
uma plataforma de gerenciamento de conteúdo que permite a criação e
manutenção de sites de forma simplificada e intuitiva.

Neste cenário o temos uma complexidade ao realizar auditorias manuais, que
consomem tempo e recursos, e principalmente a falta de ferramentas
centralizadas dentro da UFJF para aplicar o Modelo de Acessibilidade em
Governo Eletrônico~\cite{emag}, conhecido como eMag, e as Diretrizes
de Acessibilidade para Conteúdo Web~\cite{wcag22}, conhecidas como
WCAG, visto que esse documentos estabelecem parâmetros técnicos para essa inclusão.

Em uma análise preliminar avaliando a presença de elementos de acessibilidade nos
sites da UFJF, foi identificado que boa parte apresentavam falhas de acessibilidade, como imagens sem texto
alternativo e baixo contraste de cores, limitando o acesso de usuários com deficiência
visual. Além disso, a falta de padronização e de um processo de auditoria contínuo
dificulta a identificação e correção dessas falhas, comprometendo a qualidade e a
usabilidade dos sites.

Diante deste desafio, este artigo propõe uma solução técnica inovadora
para o contexto da UFJF, baseada em um plugin WordPress de auditoria
de acessibilidade que integra tecnologias modernas de automação,
análise técnica e processamento de dados. O sistema opera como um serviço
independente, com suporte a regras WCAG 2.1/2.2 e eMAG, permitindo
avaliações em tempo real e personalização de regras de acessibilidade.

A ferramenta não apenas otimiza processos técnicos, mas democratiza
a fiscalização de acessibilidade, empoderando conteudistas não especialistas
com dados claros e de fácil identificação. Este trabalho visa, portanto, contribuir
para o debate sobre automação e inclusão digital, sugerindo um modelo que buscam alinhar-se às exigências legais e éticas da
acessibilidade web, dentro do escopo do eMag, WCAG 2.1/2.2 e o contexto de cada
organização.

\subsection{Metodologia}\label{subsec:metodologia}
Diante do contexto apresentado a seguinte metodologia proposta neste trabalho
para a implementação do sistema de auditoria de acessibilidade, é dividida
em três etapas principais: análise de requisitos, onde são identificadas as
necessidades da UFJF e feitos testes com ferramentas de acessibilidade já
existentes; desenvolvimento do plugin, onde são definidas as tecnologias à serem
utilizadas, a arquitetura do sistema, a implementação das funcionalidades e a
integração com o WordPress Multi-Sites~\autocite{wp-ms}; e por fim, a avaliação da ferramenta,
onde são feitos testes de usabilidade e de acessibilidade, com o objetivo de
validar a eficácia do plugin.

Fragmentando a metodologia em etapas, é possível garantir um desenvolvimento
mais organizado e eficiente, com foco na qualidade e na usabilidade do sistema.
Além disso, a divisão em etapas permite a identificação de possíveis problemas
e a realização de ajustes ao longo do processo, garantindo a entrega de um
produto final que atenda às expectativas e necessidades da UFJF\@.

\subsubsection{Análise de Requisitos}
A primeira etapa consiste na análise dos requisitos do sistema, com base nas
necessidades da UFJF e nas diretrizes de acessibilidade. Para isso, a partir
de reuniões com a equipe de TI e conteudistas, chegou-se a um conjunto de
funcionalidades essenciais para o plugin, como: integração com o WordPress
Multi-Sites, avaliação em tempo real, suporte as diretrizes de acessibilidade,
relatórios claros e simples, personalização de regras a serem avaliadas e a
possibilidade de visualização de erros e sugestões de correção.

Como ja citado, ter o suporte as regras WCAG 2.1/2.2 e eMAG é fundamental para
garantir a conformidade dos sites da UFJF com as normas de acessibilidade.
Além disso, a utilização de um sistema como o CMS WordPress dentro da UFJF
é uma realidade, e portanto, a integração do plugin de acessibilidade com
essa plataforma não só é desejável, como também é essencial para garantir
que os relatórios possam ser facilmente compreendidos, visto que os
conteudistas da UFJF responsáveis pela atualização dos sites, muitas
vezes não possuem conhecimento técnico em relação à elementos HTML,
CSS e JavaScript, o que torna-se relevante a disponibilização dos
relatórios de forma clara e simples, e facilmente seguidos para a
correção dos erros.

É importante notar que o WordPress permite a padronização de layouts,
e certos elementos das páginas são gerenciados pelos temas desenvolvidos pelos
desenvolvedores da UFJF, portanto, tais elementos não são passíveis de alteração
pelos conteudistas, o que torna a avaliação de acessibilidade desses elementos
não necessária para o conteudista, sendo assim, o plugin deve permitir a
personalização tanto das regas a serem avaliadas, quanto dos elementos.

Uma API REST~\autocite{api}, é uma interface de programação de aplicações, que
permite a comunicação entre sistemas, e é amplamente utilizada para integração
de sistemas e serviços. A utilização de uma API REST somente para a geração
dos relatórios em tempo real, sem a necessidade de armazenamento, é uma
solução mais eficiente e escalável, visto que a UFJF possui uma quantidade
bastante significativa de sites (Inserir Valor), e armazenar todos os
relatórios de acessibilidade em um banco de dados, é tarefa que consumiria
muitos recursos e espaço em disco, tornando inviável tal abordagem.

Em fase preliminar ao desenvolvimento do projeto, realizamos testes com o
AccessMonitor~\autocite{AM},
uma ferramenta de auditoria de acessibilidade online, que permite a avaliação de
sites em tempo real. Embora a solução ofereça uma interface intuitiva para
avaliação pontual de páginas web, identificamos limitações significativas para o
contexto operacional da UFJF. As principais restrições estavam na ausência de
mecanismos de personalização de regras, impedindo a adaptação a contextos
específicos, e a escalabilidade restrita, com um limite operacional páginas a
serem avaliadas por um endereço IP\@. Adicionalmente, a necessidade de inserção
manual de URLs por avaliação tornava inviável a análise de grandes portfólios de
sites\@.

Testamos também a utilização do QualWeb~\autocite{qualweb}, que é a ferramenta
de avaliação portrás do AccessMonitor, porém, esbarramos em limitações técnicas,
semelhantesàs encontradas no AccessMonitor, e principalmente, na falta de suporte
na traduçãode relatórios para o português, o que dificultaria a compreensão dos
conteudistas da UFJF\@.

Diante dessas limitações, optamos por desenvolver um plugin de acessibilidade
personalizado, que atendesse às necessidades específicas da UFJF e que fosse
compatível com a infraestrutura tecnológica existente. Para auxiliar no
desenvolvimento da API REST que gera os relatórios de acessibilidade,
buscamos por ferramentas e tecnologias que avaliasse a acessibilidade
do conteúdo DOM~\autocite{DOM} de uma página web.

Encontramos o IBM Equal Access Toolkit~\autocite{IBMa}, uma ferramenta de código
aberto desenvolvida pela IBM, que permite a avaliação de acessibilidade de páginas
web em tempo real, com suporte as diretrizes de acessibilidade. A ferramenta oferece
o accessibility-checker~\autocite{AC} um pacote Node~\autocite{Node} que permite a
testes automatizados de acessibilidade. o pacote permite também a personalização de
regras a serem avaliadas,
porém, a dificuldade de integração com a nossa API, e a falta de suporte para o
português, impediu de seguirmos com essa ferramenta.

A ferramenta da qual decidimos utilizar foi o axe-core~\autocite{axecore}, a quarta
ferramenta de avaliação de acessibilidade que testamos. O axe-core é uma biblioteca
de código aberto, desenvolvida pela Deque Systems, que permite a avaliação de acessibilidade
de páginas web. O ponto chave que nos fez optar por essa ferramenta, foi a facilidade
de integração com a nossa API REST, a possibilidade de personalização de regras a serem
avaliadas, e o suporte para o português, o que facilitaria a compreensão dos conteudistas
da UFJF\@.

\subsubsection{Desenvolvimento do Plugin}
A segunda etapa consiste no desenvolvimento do plugin de acessibilidade, que
integra tecnologias modernas de automação, análise técnica e processamento de
dados. Para isso, optamos por utilizar a linguagem de programação JavaScript, em
em conjunto com o framework Express.js~\autocite{express}, que é um framework web
para Node.js, que permite a criação de aplicações web de forma rápida
e eficiente.

Para entendermos a integração com o WordPress Multi-Sites, prescisamos entender
o conceito de Endpoints, Rotas e o que significa Multi-Sites no contexto do WordPress.
Um endpoint~\autocite{endpoints} é uma URL específica que é usada para acessar
recursos em uma API, como por exemplo, uma lista de posts ou páginas de um site. Uma
rota~\autocite{routes} é um caminho específico que é usado para acessar um endpoint,
e é definida no arquivo de rotas da aplicação. O WordPress Multi-Sites é uma funcionalidade
do WordPress que permite a criação de vários sites a partir de uma única instalação,
e cada site possui um ID único e um domínio específico, além de possuir seu próprio
conjunto de posts e páginas.

Partindo desse entendimento, para o desenvolvimento do sistema, o
dividimos em duas partes, o plugin no gerenciador de conteúdo WordPress,e a
API REST hospedada em um servidor externo. O plugin é responsável por
consultar a API REST, e exibir os relatórios de acessibilidade na interface do
WordPress, enquanto a API REST é responsável por receber as requisições
do plugin, avaliar a acessibilidade das páginas web e retornar os relatórios
para o plugin.

A arquitetura da API REST é composta por quatro camadas principais: a camada
de roteamento a camada de autorização e autenticação, a camada de controle,
e a camada de serviços.

A camada de roteamento é responsável por definir as rotas da API, que
são os endpoints que podem ser acessados pelo plugin. As rotas são definidas
no arquivo de rotas da aplicação, e cada rota é associada a um método
HTTP~\autocite{HTTP}, como GET, POST, PUT ou DELETE. Para
podermos entendermos melhor o funcionamento das rotas, precisamos entender
o que a camada de autorização e autenticação faz.

Cada requisição feita ao plugin é autenticada e autorizada por tal camada.
A API possui a rota de login, que é responsável por receber as credenciais
necessárias para autenticar o usuário, e utilizando o conceito de JWT~\autocite{JWT},
a API gera e retorna um token de acesso, que é utilizado para autorizar as
requisições feitas pelo plugin. Antes de cada requisição ser processada pela
API, o token de acesso é verificado por um middleware~\autocite{mw},
que garante que apenas tokens válidos possam acessar as rotas protegidas.

Revisitando a camada de roteamento, é possível observar que, embora
existam múltiplas rotas disponíveis, as mais relevantes são aquelas dedicadas
à avaliação de acessibilidade: uma destinada a analisar uma página ou post
específico e outra voltada para a verificação integral de todos os posts de um
site. Ambas compartilham um elemento central em sua estrutura: o ID do
site, que atua como parâmetro obrigatório e chave para identificar o contexto
da análise. Além disso, seguem um núcleo operacional comum, no qual são
acionados os processos de geração de relatórios detalhados de acessibilidade,
garantindo consistência e padronização nos resultados.

A distinção entre as duas rotas reside em um detalhe fundamental: a rota
de avaliação de um post específico exige um complemento informativo — a
URL da página em questão, que será discutida em detalhes quando abordarmos
a camada de servços. Por outro lado, a rota de avaliação de todos os posts
de um site opera de forma mais abrangente, percorrendo todas as páginas
disponíveis e gerando relatórios. Essa diferença de escopo é essencial para
atender às demandas específicas dos conteudistas e dos desenvolvedores da UFJF\@.

A camada de controle funciona como o núcleo organizador do sistema,
responsável por intermediar a comunicação entre as requisições enviadas pelo
plugin e a execução das operações necessárias para avaliar a acessibilidade.
Ela inicia seu processo ao receber e interpretar as solicitações, direcionando-as
de forma estratégica aos serviços adequados, garantindo que cada etapa do
processamento ocorra com eficiência. Além de definir qual serviço deve ser
acionado em cada contexto, essa camada também estrutura o fluxo de dados,
determinando como eles serão validados, transformados ou enriquecidos antes
de serem enviados de volta ao plugin.


\nocite{*}
\printbibliography[title={Referências}]

\end{document}