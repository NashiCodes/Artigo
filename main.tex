\documentclass[12pt]{article}
\usepackage{xurl}
\usepackage[utf8]{inputenc}
\usepackage{csquotes}
\usepackage[brazil]{babel}
\usepackage{graphicx}
\usepackage[
    backend=biber,
    language=brazil,
    ]{biblatex}

\DeclareFieldFormat{url}{Disponível em\addcolon\space\url{#1}}
\addbibresource{references.bib}


\title{Plugin de acessibilidade para sites Governamentais}
\author{João Victor Pereira dos Anjos}
\date{Jan 2025}


\begin{document}

\maketitle
  
\tableofcontents

\section{Resumo}
   
This is the first section.
      
Lorem  ipsum  dolor  sit  amet,  consectetuer  adipiscing  
elit.   Etiam  lobortisfacilisis sem.  Nullam nec mi et 
neque pharetra sollicitudin.  Praesent imperdietmi nec ante. 
Donec ullamcorper, felis non sodales..

\section{Introdução}

A palavra acessibilidade, sua origem etimológica é derivada do 
latim \textit{accessiblitas} e significa ``condição para utilização, com segurança e autonomia, total ou assistida, dos espaços, mobiliários e equipamentos urbanos, das edificações, dos serviços de transporte e dos dispositivos, sistemas e meios de comunicação e informação, por pessoa com deficiência ou mobilidade reduzida''\cite{Acessibilidade}. No âmbito digital a acessibilidade web é um pilar fundamental para a inclusão, garantindo que todos os usuários, independentemente de suas capacidades físicas ou cognitivas, possam acessar, compreender e interagir com conteúdo online.\cite{wcag21}. 

\subsection{Contexto Institucional e Desafios Operacionais}\label{subsec:contexto}
Trazendo para o contexto brasileiro, o Modelo de Acessibilidade em Governo Eletrônico\cite{emag} e as Diretrizes de Acessibilidade para Conteúdo Web\cite{wcag22} estabelecem parâmetros técnicos para essa inclusão. No entanto, a implementação prática dessas diretrizes enfrenta desafios significativos, especialmente em instituições públicas de grande porte, A Universidade Federal de Juiz de Fora (UFJF) gerencia uma ampla rede de sites institucionais atualizados por conteudistas sem formação técnica em acessibilidade\cite{emag}. Os principais desafios incluem:

\begin{itemize}
\item Falta de padronização em elementos críticos como texto alternativo
\item Dependência de auditorias manuais demoradas
\item Dificuldade de escalabilidade na plataforma WordPress
\end{itemize}

\subsection{Diretrizes e Fundamentação Técnica}\label{subsec:diretrizes}
O sistema foi estruturado com base em:

\begin{itemize}
\item\cite{wcag22} (Níveis A e AA)
\item\cite{emag} para o setor público brasileiro
\item Framework\cite{axecore} para análise técnica
\end{itemize}

\section{Arquitetura do Sistema}\label{sec:arquitetura}
Sistema desenvolvido com arquitetura modular integrando:

\subsection{Componentes Técnicos e Funcionalidades}\label{subsec:componentes}
\begin{itemize}
\item Renderização headless com Puppeteer\cite{puppeteer}
\item Motor de análise\cite{axecore} (203 regras WCAG)
\item Módulo de pós-processamento Cleaner.js
\end{itemize}

\subsection{Integração com WordPress e APIs REST}\label{subsec:api}
\begin{itemize}
\item Autenticação via\cite{jwt}
\item Endpoints REST para avaliação de conteúdo
\item Plugin WordPress com interface intuitiva
\end{itemize}

\nocite{*}
\printbibliography[title={Referências}]

\end{document}