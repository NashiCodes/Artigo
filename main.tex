\documentclass[12pt]{article}
\usepackage[utf8]{inputenc}
\usepackage{csquotes}
\usepackage[brazil]{babel}
\usepackage{graphicx}
\usepackage[
    backend=biber,
    style=abnt,
    language=brazil,
    ]{biblatex}

\addbibresource{references.bib}


\title{Plugin de acessibilidade para sites Governamentais}
\author{João Victor Pereira dos Anjos}
\date{Jan 2025}


\begin{document}

\maketitle
  
\tableofcontents

\section{Resumo}
   
This is the first section.
      
Lorem  ipsum  dolor  sit  amet,  consectetuer  adipiscing  
elit.   Etiam  lobortisfacilisis sem.  Nullam nec mi et 
neque pharetra sollicitudin.  Praesent imperdietmi nec ante. 
Donec ullamcorper, felis non sodales..

\section{Introdução}

A palavra acessibilidade, sua origem etimológica é derivada do latim \textit{accessiblitas} 
e significa ``condição para utilização, com segurança e autonomia, 
total ou assistida, dos espaços, mobiliários e equipamentos urbanos, das 
edificações, dos serviços de transporte e dos dispositivos, sistemas e meios de
comunicação e informação, por pessoa com deficiência ou mobilidade reduzida''\cite{Acessibilidade}.

No Brasil, a acessibilidade é um direito garantido pela Constituição 
Federal de 1988, pela Lei Brasileira de Inclusão (LBI) de 2015 \Cite{LBI}
e por normas técnicas específicas, como a NBR 9050/2015 da Associação 
Brasileira de Normas Técnicas~\cite{ABNT}. Essas legislações estabelecem 
parâmetros para a promoção da acessibilidade em espaços públicos e privados,
visando a inclusão de pessoas com deficiência física, visual, auditiva, 
intelectual e múltipla.

No âmbito digital a acessibilidade web é um pilar fundamental para a 
inclusão, garantindo que todos os usuários, independentemente de suas
capacidades físicas ou cognitivas, possam acessar, compreender e interagir com
conteúdo online.\cite{wcag22}. 

A Universidade Federal de Juiz de Fora (UFJF), como uma instituição pública,
gerência uma grande quantidade de sites e portais, que são regularmente
atualizados por diversas pessoas, como professores, pesquisadores, bolsistas e
servidores, da qual chamamos de conteudistas. A diversidade de conteúdos e
responsáveis torna o processo de garantia de acessibilidade primordial para
atender a legislação e promover a inclusão digital.

Neste cenário o temos uma complexidade ao realizar auditorias manuais, que 
consomem tempo e recursos, e principalmente a falta de ferramentas 
centralizadas dentro da UFJF para aplicar o Modelo de Acessibilidade em 
Governo Eletrônico~\cite{emag}, conhecido como eMag, e as Diretrizes 
de Acessibilidade para Conteúdo Web~\cite{wcag22}, conhecidas como 
WCAG, visto que esse documentos estabelecem parâmetros técnicos para essa inclusão.

Em uma análise preliminar avaliando a presença de elementos de acessibilidade nos
sites da UFJF, foi identificado que boa parte apresentavam falhas de acessibilidade, como imagens sem texto
alternativo e baixo contraste de cores, limitando o acesso de usuários com deficiência
visual. Além disso, a falta de padronização e de um processo de auditoria contínuo
dificulta a identificação e correção dessas falhas, comprometendo a qualidade e a
usabilidade dos sites.

Diante deste desafio, este artigo propõe uma solução técnica inovadora para o contexto
da UFJF, baseada em um plugin WordPress de acessibilidade que integra tecnologias
modernas de automação, análise técnica e processamento de dados. O sistema opera
como um serviço independente, com suporte a regras WCAG 2.1/2.2 e eMAG, e é
compatível com a API REST do WordPress, permitindo avaliações em tempo real e
personalização de regras de acessibilidade.

A ferramenta não apenas otimiza processos técnicos, mas democratiza 
a fiscalização de acessibilidade, empoderando conteudistas não especialistas 
com dados claros e de fácil identificação. Este trabalho visa, portanto, contribuir
para o debate sobre automação e inclusão digital, sugerindo um modelo que buscam alinhar-se às exigências legais e éticas da
acessibilidade web, dentro do escopo do eMag, WCAG 2.1/2.2 e o contexto de cada
organização.

\subsection{Metodologia}\label{subsec:metodologia}
A fim de divulgar as informações referentes aos seus setores e atividades a Universidade Federal de Juiz de Fora, UFJF, através do Centro de Gestão do Conhecimento Organizacional, CGCO, é responsável por controlar a disponibilização de sites, a padronização dos layouts e o suporte técnico. Para a sustentação desse serviço, é utilizado o CMS WordPress.

Diante deste contexto, torna-se fundamental a implementação de um sistema de
auditoria automatizada de acessibilidade, que possa ser integrado ao WordPress e
avaliar continuamente a conformidade dos sites com as diretrizes de acessibilidade
WCAG 2.1/2.2 e eMAG\@. Para isso, a metodologia proposta neste trabalho é dividida
em três etapas principais: análise de requisitos, desenvolvimento do plugin e
avaliação da ferramenta.

\subsubsection{Análise de Requisitos}
A primeira etapa consiste na análise dos requisitos do sistema, com base nas
necessidades da UFJF e nas diretrizes de acessibilidade WCAG 2.1/2.2 e eMAG\@. Para
isso, a partir de reuniões com a equipe de TI e conteudistas, chegou-se a um
conjunto de funcionalidades essenciais para o plugin, como: integração com o
WordPress, avaliação em tempo real, suporte a regras WCAG 2.1/2.2 e eMAG, relatórios
claros e simples, personalização de regras a serem avaliadas e a possibilidade de
visualização de erros e sugestões de correção.

\nocite{*}
\printbibliography[title={Referências}]

\end{document}