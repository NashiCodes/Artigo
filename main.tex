\documentclass[12pt]{article}
\usepackage[utf8]{inputenc}
\usepackage{csquotes}
\usepackage[brazil]{babel}
\usepackage{graphicx}
\usepackage[
    backend=biber,
    style=abnt,
    language=brazil,
    ]{biblatex}

\addbibresource{references.bib}


\title{Plugin de acessibilidade para sites Governamentais}
\author{João Victor Pereira dos Anjos}
\date{Jan 2025}


\begin{document}

\maketitle
  
\tableofcontents

\section{Resumo}
   
This is the first section.
      
Lorem  ipsum  dolor  sit  amet,  consectetuer  adipiscing  
elit.   Etiam  lobortisfacilisis sem.  Nullam nec mi et 
neque pharetra sollicitudin.  Praesent imperdietmi nec ante. 
Donec ullamcorper, felis non sodales..

\section{Introdução}

A palavra acessibilidade, sua origem etimológica é derivada do latim \textit{accessiblitas} 
e significa ``condição para utilização, com segurança e autonomia, 
total ou assistida, dos espaços, mobiliários e equipamentos urbanos, das 
edificações, dos serviços de transporte e dos dispositivos, sistemas e meios de
comunicação e informação, por pessoa com deficiência ou mobilidade reduzida''\cite{Acessibilidade}.

No Brasil, a acessibilidade é um direito garantido pela Constituição 
Federal de 1988, pela Lei Brasileira de Inclusão (LBI) de 2015 \Cite{LBI}
e por normas técnicas específicas, como a NBR 9050/2015 da Associação 
Brasileira de Normas Técnicas~\cite{ABNT}. Essas legislações estabelecem 
parâmetros para a promoção da acessibilidade em espaços públicos e privados,
visando a inclusão de pessoas com deficiência física, visual, auditiva, 
intelectual e múltipla.

No âmbito digital a acessibilidade web é um pilar fundamental para a 
inclusão, garantindo que todos os usuários, independentemente de suas
capacidades físicas ou cognitivas, possam acessar, compreender e interagir com
conteúdo online.\cite{wcag22}. 

A Universidade Federal de Juiz de Fora (UFJF), como uma instituição pública,
gerência uma grande quantidade de sites e portais, que são regularmente
atualizados por diversas pessoas, como professores, pesquisadores, bolsistas e
servidores, da qual chamamos de conteudistas. A diversidade de conteúdos e
responsáveis torna o processo de garantia de acessibilidade primordial para
atender a legislação e promover a inclusão digital.

Neste cenário o temos uma complexidade ao realizar auditorias manuais, que 
consomem tempo e recursos, e principalmente a falta de ferramentas 
centralizadas dentro da UFJF para aplicar o Modelo de Acessibilidade em 
Governo Eletrônico~\cite{emag}, conhecido como eMag, e as Diretrizes 
de Acessibilidade para Conteúdo Web~\cite{wcag22}, conhecidas como 
WCAG, visto que esse documentos estabelecem parâmetros técnicos para essa inclusão.

Em uma análise preliminar avaliando a presença de elementos de acessibilidade nos
sites da UFJF, foi identificado que boa parte apresentavam falhas de acessibilidade, como imagens sem texto
alternativo e baixo contraste de cores, limitando o acesso de usuários com deficiência
visual. Além disso, a falta de padronização e de um processo de auditoria contínuo
dificulta a identificação e correção dessas falhas, comprometendo a qualidade e a
usabilidade dos sites.

Diante deste desafio, este artigo propõe uma solução técnica inovadora para o contexto
da UFJF, baseada em um plugin WordPress de acessibilidade que integra tecnologias
modernas de automação, análise técnica e processamento de dados. O sistema opera
como um serviço independente, com suporte a regras WCAG 2.1/2.2 e eMAG, e é
compatível com a API REST do WordPress, permitindo avaliações em tempo real e
personalização de regras de acessibilidade.

A ferramenta não apenas otimiza processos técnicos, mas democratiza 
a fiscalização de acessibilidade, empoderando conteudistas não especialistas 
com dados claros e de fácil identificação. Este trabalho visa, portanto, contribuir
para o debate sobre automação e inclusão digital, sugerindo um modelo que buscam alinhar-se às exigências legais e éticas da
acessibilidade web, dentro do escopo do eMag, WCAG 2.1/2.2 e o contexto de cada
organização.

\subsection{Metodologia}\label{subsec:metodologia}
A fim de divulgar as informações referentes aos seus setores e atividades a 
Universidade Federal de Juiz de Fora, UFJF, através do Centro de Gestão 
do Conhecimento Organizacional (CGCO), é responsável por controlar a 
disponibilização de sites, a padronização dos layouts e o suporte técnico. Para
a sustentação desse serviço, é utilizado o CMS WordPress~\footfullcite{WP}.

Diante deste contexto, torna-se fundamental a implementação de um sistema de
auditoria automatizada de acessibilidade, que possa ser integrado ao WordPress e
avaliar continuamente a conformidade dos sites com as diretrizes de acessibilidade
WCAG 2.1/2.2 e eMAG\@. Para isso, a metodologia proposta neste trabalho é dividida
em três etapas principais: análise de requisitos, desenvolvimento do plugin e
avaliação da ferramenta.

\subsubsection{Análise de Requisitos}
A primeira etapa consiste na análise dos requisitos do sistema, com base nas
necessidades da UFJF e nas diretrizes de acessibilidade WCAG 2.1/2.2 e 
eMAG\@. Para isso, a partir de reuniões com a equipe de TI e conteudistas, 
chegou-se a um conjunto de funcionalidades essenciais para o plugin, como: 
integração com o WordPress Multi-Sites, avaliação em tempo real, suporte a 
regras WCAG 2.1/2.2 e eMAG, relatórios claros e simples, personalização de 
regras a serem avaliadas e a possibilidade de visualização de erros e sugestões 
de correção.

O eMag é um modelo de acessibilidade em governo eletrônico que estabelece
diretrizes para a promoção da acessibilidade em sites governamentais, com o
objetivo de garantir a inclusão digital e o acesso à informação para todos os
cidadãos. Já o WCAG 2.1/2.2 é um conjunto de diretrizes internacionais para a
acessibilidade de conteúdo web, que visa tornar os sites mais acessíveis para
pessoas com deficiência visual, auditiva, motora e cognitiva. Portanto, ter o
suporte a essas regras é fundamental para garantir a conformidade dos sites da
UFJF com as normas de acessibilidade.

Os conteudistas da UFJF responsáveis pela atualização dos sites, muitas vezes não
possuem conhecimento técnico em relação à elementos HTML, CSS e JavaScript, o que
torna-se relevante a disponibilização de relatórios claros e simples, que possam
ser facilmente compreendidos e seguidos para a correção dos erros de acessibilidade,
e a possibilidade de visualizar os erros e sugestões de correção diretamente no
painel do WordPress.

A utilização de um sistema como o CMS WordPress dentro da UFJF é uma realidade, e
portanto, a integração do plugin de acessibilidade com essa plataforma não só
é desejável, como também é essencial para garantir que os relatórios sejam
facilmente acessíveis e que as correções possam ser feitas de forma rápida e
eficiente. É importante notar que o WordPress permite a padronização de layouts,
e certos elementos das páginas são gerenciados pelos temas desenvolvidos pelos
desenvolvedores da UFJF, portanto, tais elementos não são passíveis de alteração
pelos conteudistas, o que torna a avaliação de acessibilidade desses elementos
não necessária para o conteudista, sendo assim, o plugin permite a personalização
tanto das regas a serem avaliadas, quanto dos elementos.

A UFJF possui uma quantidade bastante significativa de sites (Inserir Valor), portais e blogs, desta
forma, salvar os relatórios de acessibilidade em um banco de dados, é uma tarefa que 
consumiria muitos recursos e espaço em disco, portanto, a utilização de uma API REST
que apenas gere os relatórios em tempo real, sem a necessidade de armazenamento, é 
uma solução mais eficiente e escalável.
 
Foi também durante a etapa de análise de requisitos, que analisamos 
outras ferramentas e tecnologias disponíveis no mercado, como plugins 
de acessibilidade já existentes para WordPress, sistemas de auditoria de acessibilidade.

Em uma fase preliminar do projeto, realizamos testes com o AccessMonitor~\footfullcite{AM}, uma ferramenta de auditoria de acessibilidade online, que permite a avaliação de sites em tempo real. Embora a solução ofereça uma interface intuitiva para avaliação pontual de páginas web, identificamos limitações significativas para o contexto operacional da UFJF, As principais restrições estavam na ausência de mecanismos de personalização de regras, impedindo a adaptação a contextos específicos, e a escalabilidade restrita, com um limite operacional páginas a serem avaliadas por um endereço IP\@. Adicionalmente, a necessidade de inserção manual de URLs por avaliação tornava inviável a análise de grandes portfólios de sites, realidade comum em instituições de porte universitário.

\nocite{*}
\printbibliography[title={Referências}]

\end{document}